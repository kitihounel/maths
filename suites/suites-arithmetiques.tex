\documentclass{beamer}
\usepackage[french]{babel}
\usepackage{default}
\usepackage{lmodern}
\usepackage{tikzsymbols}
\usepackage{hyperref}
\usepackage{ragged2e}
\usepackage{amsmath}

% Allow text to be justified on slides.
% https://tex.stackexchange.com/questions/55589/text-justify-in-beamer/475806
% \usepackage{ragged2e}
% \apptocmd{\frame}{}{\justifying}{}
% Much simpler.
% https://latex.org/forum/viewtopic.php?t=2722
\justifying

\usetheme{metropolis}

\title{Suites arithmétiques}
\author{}
\date{}

\begin{document}
\setbeamercovered{transparent} 
\begin{frame}
  \titlepage
\end{frame}

\begin{frame}{Avant de commencer}
  J'avais annoncé que de temps en temps je parlerais de toutes ces choses qu'on a étudiées lors des cours de maths sans vraiment en connaître l'utilité. Et de prouver que les maths sont pas si difficiles et qu'il est toujours bon de les connaître, même un peu.

  On commence donc en douceur avec les \textbf{suites numériques}, plus précisément les \textbf{suites arithmétiques}.

  \begin{center}
    \Cooley[2][cyan]
  \end{center}
\end{frame}

\begin{frame}{Définition}
  Une suite numérique est une fonction définie uniquement sur des nombres entiers positifs $0, 1, 2, \dots$\ C'est un ensemble de valeurs obtenues en appliquant une fonction à des entiers. Les différentes valeurs qui constituent la suite sont appelées \textbf{termes} de la suite.

  Les suites numériques sont généralement nommées u, v, etc.

  Une suite numérique peut être construite de deux manières. Avec une formule explicite de calcul de ses éléments ou à l'aide d'une formule de récurrence.
\end{frame}

\begin{frame}{Définition à l'aide d'une formule explicite}
  C'est le cas le plus simple. Il y a une formule pour calculer la valeur d'un terme donné de la suite. Par exemple,
  \begin{align*}
    u_n &= 3 * n + 5 \\
    v_n &= \frac{1}{n+2} - 7.   
  \end{align*}

  Ainsi on peut calculer
  \begin{gather*}
    u_0 = 3 * 0 + 5 = 5 \\
    u_4 = 3 * 4 + 5 = 17
  \end{gather*}

  \begin{gather*}
    v_1 = \frac{1}{1+2} - 7 = \frac{1}{3} - 7 = -6,66 \\
    v_3 = \frac{1}{3+2} - 7 = \frac{1}{5} - 7 = -6,8
  \end{gather*}
\end{frame}

\begin{frame}{Définition à l'aide d'une formule de récurrence}
  Dans ce cas, on a la valeur du premier terme de la suite et on sait comment calculer la valeur d'un terme à partir du terme précédent. Par exemple,
  \[
    u_n :
    \begin{cases}
      u_0 &= 1 \\
      u_{n+1} &= 2 * u_{n-1} + 7.
    \end{cases}
  \]
  
  On pourra donc calculer les autres termes de la suite de la manière suivante.
  \begin{align*}
    u_1 & \ = & 2 * u_0 + 7 & \ = & 2 * 1 + 7  & \ = & 2 + 7  & \ = & 9  \\
    u_2 & \ = & 2 * u_1 + 7 & \ = & 2 * 9 + 7  & \ = & 18 + 7 & \ = & 25 \\
    u_3 & \ = & 2 * u_2 + 7 & \ = & 2 * 25 + 7 & \ = & 20 + 7 & \ = & 57 \\
    \dots & & & & & & & &
  \end{align*}
\end{frame}

\begin{frame}{Suites arithmétiques}
  On rencontre souvent deux types particulières de suites : les suites arithmétiques et les suites géométriques. On traitera ici uniquement des suites arithmétiques. Les suites géométrique feront l'objet d'une autre série.

  Une suite arithmétique est une suite où la valeur d'un terme est calculée en ajoutant une valeur constante appelée \textbf{raison} au terme précédent. Une suite arithmétique est caractérisée par deux éléments : le premier terme et la raison.
  
  \[
    u_n:
    \begin{cases}
      u_0 &= x \\
      u_{n+1} &= u_{n-1} + r
    \end{cases}
  \]
\end{frame}

\begin{frame}{Exemples de suites arithmétiques I}
  \[
    u_n =
    \begin{cases}
      11 & \ si\ n\ =\ 0 \\
      u_{n-1} - 2,5 &\ si\ n\ >\ 0
    \end{cases}
  \]
  
  \[
    v_n =
    \begin{cases}
      2 & \ si\ n\ =\ 0 \\
      v_{n-1} + 3 &\ si\ n\ >\ 0
    \end{cases}
  \]
  
  Les premiers termes de la suite $v$ sont
  \begin{center}
    $2, 5, 8, 11, 14, 17, 20, 23, \dots$
  \end{center}

  La suite $u$ a une raison égale à $-2,5$ et la suite $v$ une raison égale à $3$. Rien n'empêche une suite d'avoir une raison qui soit un nombre réel ou un nombre négatif.
\end{frame}

\begin{frame}{Exemples de suites arithmétiques II}
On ne s'en rend pas toujours compte, mais on utilise couramment des suites arithmétiques.

L'ensemble des entiers naturels $\mathbb{N} = \{0, 1, 2, \dots\}$ forme une suite arithmétique de premier terme $0$ et de raison $1$.

L'ensemble des nombres pairs $\mathbb{P} = \{0, 2, 4, 6, \dots\}$ forme une suite arithmétique de premier terme $0$ et de raison $2$.

L'ensemble des nombres impairs $\mathbb{I} = \{1, 3, 5, 7, \dots\}$ forme une suite arithmétique de premier terme $1$ et de raison $2$.
\end{frame}

\begin{frame}{Formule de calcul des termes d'une suite arithmétique}
On a vu que les suites arithmétiques étaient définies par une formule de récurrence. Et quand une suite est définie de cette manière, si l'on veut calculer la valeur d'un terme, il faut connaître la valeur du terme précédent.

Donc si je veux calculer la valeur de $u_{10}$, il me faut connaître la valeur de $u_9$ qui requiert à son tour de connaître la valeur de $u_8$, et ainsi de suite. Mais heureusement, dans le cas des suites arithmétiques, il existe une formule permettant de calculer directement la valeur d'un terme. \Smiley

Et cette formule se déduit de façon toute simple.
\end{frame}

\begin{frame}{Déduction de la formule de calcul}
Pour trouver la formule, il faut juste se rappeler de comment est calculée la valeur d'un terme  à partir du terme précédent.

\begin{displaymath}
  \begin{array}{lllllll}
    u_1 & = & u_0 + r &   & 				 	    &   &					 \\
    u_2 & = & u_1 + r & = & u_0 + \ r + r & = & u_0 + 2r \\
    u_3 & = & u_2 + r & = & u_0 + 2r  + r & = & u_0 + 3r \\
    u_4 & = & u_3 + r & = & u_0 + 3r  + r & = & u_0 + 4r
  \end{array}
\end{displaymath}

On observe donc pour une suite arithmétique dont on connaît le premier terme $u_0$ et la raison $r$, on a la propriété :
\[
  u_n = u_0 + nr,\ \forall n
\]
\end{frame}

\begin{frame}{Utilité des suites arithmétiques I}
  Supposons que vous ouvriez un compte à intérêts simples dans une banque. Le taux d'intérêt est de 5\% et vous déposez $10\ 000\ 000$ dans le compte. Donc à la fin de chaque année, vous recevrez des intérêts s'élevant à $500\ 000$.

  Après un an, le solde du compte passera à $10\ 500\ 000$. Après deux ans, vous disposerez de $11\ 000\ 000$. Au bout de la troisième année le solde sera de $11\ 500\ 000$. On constate donc que le solde de votre compte à la fin d'une année donnée n'est rien que le solde à la fin de l'année précédente plus les $500\ 000$ d'intérêts.
\end{frame}

\begin{frame}{Utilité des suites arithmétiques II}
  Le solde de votre compte d'année en année suit donc une suite arithmétique dont le premier terme est le montant du dépôt initial $10\ 000\ 000$ et dont la raison est le montant des intérêts $500\ 000$.

  \[
    \begin{cases}
      u_0 &= 10\ 000\ 000 \\
      u_{n+1} &= u_{n-1} + 500\ 000
    \end{cases}
  \]
\end{frame}

\begin{frame}{Utilité des suites arithmétiques III}
  Les suites arithmétiques peuvent donc servir à modéliser l'évolution des comptes bancaires à intérêts simples.

  Mais elles ont d'autres applications. En fait on retrouve des suites arithmétiques dans les phénomènes quotidiens. Tous les évènements qui se produisent à intervalles réguliers forment des suites arithmétiques. Les marches des escaliers ont des hauteurs régulièrement espacées si le maçon qui les a bâties est consciencieux, vos dates d'anniversaire sont espacées de 365 jours (sauf si vous êtes né un 29 février), les graduations sur les règles sont espacées de 1mm, etc. Même les tailles des frères Dalton forment une suite arithmétique.

  Regardez autour de vous et observez attentivement. Vous trouverez d'autres exemples. Tout ce qui se produit à intervalle régulier définit une suite arithmétique.
\end{frame}

\begin{frame}{Petit exercice de fin}
Supposons que vous ayez envie de savoir dans combien d'années le solde de votre compte atteindra un certain montant. Par exemple, vous avez besoin de $12\ 800\ 000$ pour vous acheter une nouvelle voiture ou une parcelle.

Si on traduit cela sous forme d'équation, cela revient à chercher le plus petit nombre entier $K$ tel que $u_K \ge 12\ 800\ 000$. On cherche donc le premier terme de notre suite qui suit supérieur ou égal à la somme désirée.

\begin{align*}
  u_n &\ge 12\ 800\ 000 \\
\end{align*}
\end{frame}

\begin{frame}{Résolution de l'équation}
\begin{align*}
  u_0 + nr 	&\ge 12\ 800\ 000 \\
  nr 				&\ge 12\ 800\ 000 - u_0 \\
  nr 				&\ge 12\ 800\ 000 - 10\ 000\ 000 \\
  nr 				&\ge 2\ 800\ 000 \\
  n 				&\ge \frac{2\ 800\ 000}{500\ 000} \\
  n 				&\ge \frac{28}{5} \\
  n 				&\ge 5,6
\end{align*}
\end{frame}

\begin{frame}{Solution de l'équation}
Il vous faudra donc attendre six ans. \Smiley
\end{frame}

\begin{frame}{Mot de fin}
Fin de la première présentation. J'espère que cela vous a plu.

Toute suggestion, idée d'amélioration ou proposition de sujets à aborder est bienvenue.

\`{A} bientôt pour une nouvelle série.

\begin{center}
  \dSmiley[3]
\end{center}
\end{frame}
\end{document}
